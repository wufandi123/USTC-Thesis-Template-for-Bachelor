\chapter{展望与总结}
\section{工作总结}
论文从深度学习入手,介绍了现有的CPU、GPU 在面对大规模的深度学习神经网络应用时的困境,从而引入了专用于深度学习神经网络的神经网络处理器。为了能让用户正确稳定的使用神经网络处理器,神经网络处理器编译器的测试工作是很重要的,同时,由于神经网络处理器内部运算部件和数据存储结构等多方面创新性的不同,一套能用于检测神经网络处理器编译器的测试框架是必要的。

在调研了传统软件以及传统编译器的测试方式后,我们结合传统软件测试的方法和神经网络处理器编译器的独特性质,设计了一套专门用于检测神经网络处理器编译器的测试框架,通过该测试框架,我们提高了测试效率。

最后,我们详细地阐述了测试框架中最重要的一环——随机神经网络生成器,解释了随机网络生成器的功能和实现方法,最后,我们搭建了一个数据库来保证测试工作的高效性并解释了数据库的设计思想。

\section{下一步研究方向}
对于神经网络处理器编译器测试工作来说,我们接下来将继续完善我们的随机网络生成器,增加能支持的操作类型。除外,我们还将继续移植其他主流编程框架、并对新的编译器进行测试。

神经网络处理器编译器的复杂性,决定了我们测试工作的复杂性;而我们测试工作的高效性,则决定了编译器乃至整个神经网络处理器的高效性,为此,我们将会不断努力。