\chapter{绪论}
\section{研究背景}
随着大数据时代的到来,计算机朝着智能化的趋势不断发展。深度学习等新兴方法已经卓有成效的应用在各个领域。特别是深度神经网络的出现,
让很多领域都取得了突破性的进展。在每年最大的数据集ImageNet图像识别大赛中,微软研究团队所设计的图像识别系统通过运用深度神经网络技术使得识别图像的正确率达到了97\%,定位正确率达到了91\%;而在国际多通道语音识别大赛上,科大讯飞团队采用深度神经网络,使得识词错误率达到了2.24\%(六麦克风),相比于原来的传统机器学习算法,其语音识别错误率下降了接近六成。 除此之外,在2016年3月举世瞩目的人机大战中,AlphaGo战胜围棋高手李世石,深度学习逐渐走上神坛,成为计算机科学中最为炙手可热的研究方向。

这场深度学习席卷袭来的应用浪潮,无疑是被硬件、芯片这几年的飞速发展和惨烈的竞争支撑起来的。

对比CPU、GPU与FPGA这三种传统处理器,传统的中央处理器CPU虽然通用性较好,但是它单次只能处理少量任务,并行性较低,不适合与神经网络的高并行性计算,对于深度学习来讲,只能起到辅助作用;GPU拥有高并行的计算的特点,且在深度学习所需求的浮点数计算上性能极其出众,在神经网络的训练阶段大放异彩。但是在庞大的需求规模下,数据中心的GPU集群的能耗相当恐怖,对移动端的神经网络前向运算实用性差。FPGA由于其可编程性的特征可以灵活适应飞速发展的神经网络算法的变迁,且性能功耗比出色,但是想达到领先GPU、CPU计算性能的高端FPGA芯片成本太高,不适合数据中心的大规模集群布置。

在这种情况下,专用语深度学习,神经网络的神经网络处理器成为了目前最热门的研究方向。神经网络处理器,是指具有模仿人的大脑判断能力和适应能力、可并行处理多种数据功能的处理器。相较于CPU、GPU,神经网络处理器结合了神经网络模型的数据局部性特点以及计算特性,进行存储体系以及专用硬件设计,从而具有更好的性能加速比以及计算功耗比。目前,国际上流行的神经网络处理器有:IBM与美国国防部共同研发的类脑芯片TrueNorth,谷歌公司推出的与编程框架Tensorflow高度契合的神经网络处理器TPU,以及中科院计算所所推出的寒武纪系列神经网络处理器(IPU)。

从2008年起,中科院计算技术研究所智能处理器研究中心开展了寒武纪系列神经网络处理器的研发,这也是国际上首个深度学习处理器结构。当前寒武纪系列已包含3种原型处理器结构:DianNao,单核神经网络处理器结构;DaDianNao,面向超大规模神经网络的多核处理器结构;PuDianNao,面向多种机器学习算法。在若干代表性深度神经网络上的实验结果表明,DianNao的平均性能超过主流CPU核的100倍,但是面积仅为1/30而功耗则仅为1/5,效能提升可达3个数量级;DianNao的平均性能与主流GPU相当,但面积和功耗仅为主流GPU百分之一量级。2016年,中科院计算所的研究团队又提出了了深度学习指令集Cambrican,试图在更为泛化的层面来完成AI加速器的设计。

为了方便用户编程,减少开发难度,我们设计了一套包括库和编译器的软件,为了保证这套软件的正确性和可靠性,我们需要对其进行测验。然而,徒手检查代码工作量大,而仅靠随机生成的指令又无法覆盖到实际使用需求的指令序列,而且对于专用的神经网络处理器来说,由于在运算部件和数据存储结构等多方面创新性的不同,导致了神经网络处理器编译器的验证方式与传统编译器很有大差别。因此,设计一个能用于检测神经网络处理器编译器的测试框架势在必行。

\section{论文的组织结构}
第一章绪论部分介绍了由于深度学习近年来的流行,更快更低功耗的硬件设备被人们所需求,这导致了神经网络处理器的出现与发展。而为了更有效可靠的使用神经网络处理器来将深度学习应用在学术研究和产品中,需要在开发过程中,对神经网络处理器编译器的正确性进行验证。

第二章相关工作部分首先介绍了传统软件的验证方法,接着介绍了传统编译器的验证手段,通过分析深度学习指令集的特性,阐明了神经网络处理器编译器与传统编译器的差别,证明了神经网络处理器编译器验证框架的重要性。

第三章首先先从神经网络处理器的整体构架说起,而后详细的介绍了神经网络处理器的软件构架并着重点出了我们验证的主要目标。接着,从prototxt的生成到caffemodel的建立,再到caffe的的重载直到最后指令的调用,层层深入,详细介绍了神经网络处理器编译器的验证流程,并简要介绍了使用该测验框架后,测验工作的进程。

第四章作为本文的重点,介绍了神经网络处理器编译器验证框架中最重要的随机网络生成器的实现。吸取了编程框架caffe中数据结构的优点并对其结构进行改良,将层的连接方式以及网络的构建融入网络生成器的创建当中,并实现了一系列方法用于满足用户对结构以及参数的要求,最后,建立了一个数据库用于保证测试的高效性,同时阐述了数据库的设计思想。

第五章总结了论文的研究工作,并针对现有的神经网络处理器编译器测试框架提出了几点改进,并展望了未来神经网络处理器验证的工作的开展及研究方向。

