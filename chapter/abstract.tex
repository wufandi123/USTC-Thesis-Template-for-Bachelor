\begin{cnabstract}
随着深度学习算法的复杂化和应用规模的不断增长,传统的计算机CPU、GPU芯片在进行神经网络运算时,受限于功耗与性能,不能满足计算的需求。我们的寒武纪深度学习处理器(以下简称神经网络处理器),通过定制专用的运算部件、设计高效的片上存储,有效地提高了处理器的运算、访存效率,从而获得相对通用处理器数量级提升的性能优势,在学术界和工业界获得了广泛关注。

为了方便用户编程,减少开发难度,一个可靠、安全、稳定的神经网络处理器编译器是不可或缺的。而为了达到这个目的,就必须要由完善的测验·流程与高效的测验框架来对其进行验证。本文针对神经网络处理器指令以及编译器的特性进行了分析,提出并实现了一套完整的测验流程,为此我们设计了随机指令产生器,随机网络产生器,寒武纪测验框架。

本文首先介绍了传统软件以及编译器的测验方法,结合深度学习指令集的特性,分析了神经网络处理器编译器与传统编译器的不同,而后针对二者的不同,我们设计了一套完整的测验流程并实现了一个的神经网络处理器编译器测验框架。最后,我们重点讨论了神经网络处理器编译器测验框架的核心部分随机神经网络生成器的设计与实现。

\keywords{深度学习\enskip 神经网络处理器\enskip Caffe编程框架\enskip 测验框架}
\end{cnabstract}

\begin{enabstract}
This is USTC thesis template for bachelor, master and doctor user's guide. The template is created by ywg@USTC and a derivative of USTC Bachelor and Master-PhD templates. Besides that
the usage of the template, a brief
guideline for writing thesis is also provided.

\enkeywords{University of Science and Technology of China (USTC), Thesis, Universal \LaTeX{} Template, Bachelor, Master, PhD}
\end{enabstract}
